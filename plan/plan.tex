\documentclass{article}

\usepackage[utf8]{inputenc}
\usepackage[T1]{fontenc}
\usepackage[portuguese]{babel}

\usepackage[inline]{enumitem}

\usepackage{listing}

\title{Programa do Minicurso}
\author{}
\date{}

\begin{document}

\maketitle

\section{Identificação}

\begin{itemize}
    \item[] Disciplina: Minicurso de Java Moderno
    \item[]
        \hspace{-1em}
        \begin{tabular}{ccc}
            Carga Horária: 18 horas-aula&
            Teóricas: 10&
            Práticas: 8
        \end{tabular}
    \item[] Período: Setembro a Novembro de 2017.
\end{itemize}

\section{Cursos (público alvo)}
\begin{itemize}
    \item[] Ciências da Computação;
    \item[] Sistemas de Informação.
\end{itemize}

\section{Requisitos}
\begin{itemize}
    \item[] Nenhum.
\end{itemize}

\section{Ementa}
\begin{itemize*}[label={}]
        % Básico
    \item Características principais da linguagem;
    \item Aritmética em programação;
    \item Estruturas Condicionais e de Seleção;
    \item Estruturas de Repetição;
    \item Métodos, Procedimentos e Funções;

        % Orientação a Objetos
    \item Tipos definidos por usuário;
    \item Conceito de Classe, Objeto e Atributo;
    \item Interfaces;
    \item Métodos \textit{default} e encapsulamento de atributos;
    \item Classes abstratas;

        % Java-specific
    \item Processo de compilação e \textit{Bytecode};
    \item \textit{Generics};
    \item Classe Anônima;
    \item Mecanismo de alocação de memória da JVM\@;
    \item Elementos estáticos;

        % Engenharia de Software
    \item Reuso;
    \item Padrões de Projeto;
    \item Diagramas UML (\textit{Unified Modeling Language}): Classes
        e Casos de Uso;
    \item Padrão de projeto MVC\@;
    \item Padrão de projeto Composição;

        % Geral
    \item Tempo de vida de variáveis;
    \item Boas Práticas de Programação (padrões de nomenclatura/casing, etc.);
    \item Desenvolvimento de APIs adequadas;
    \item Tratamento de Exceções;
    \item \textit{Callback};
    \item Funções \textit{lambda};

        % Otimização
    \item \textit{Cache friendliness};
    \item \textit{JIT Compilation};
    \item Mecanismo de otimização da JVM\@;
    \item Disco rígido: \textit{Disk buffer};
    \item Algoritmos de escalonamento de disco;

        % Ciência da Computação
    \item Algoritmos de busca e ordenação;
    \item Análise de Algoritmos e Complexidade Assintótica;
    \item Complexidade amortizada;

        % Limbo of decisions
    \item \textit{HashMap};
    \item \textit{Streams};
    \item Serialização;

        % Eclipse
    \item Ferramenta de desenvolvimento Eclipse;
    \item Depuração;

        % Estruturas de dados
    \item Listas e Pilhas.
\end{itemize*}

\section{Objetivos}
\begin{description}
    \item[Geral:] Auxiliar na aprendizagem das disciplinas de Programação
        Orientada a Objetos 1 e 2, produzindo um projeto em console Java,
        mostrando normas e passos para um código bem feito e reutilizável.
    \item[Específico:] São objetivos do minicurso:
        \begin{itemize}[label={-}]
            \item Apresentar as especificidades básicas da linguagem Java,
                dentre elas:
                \begin{itemize}[label={-}]
                    \item Classe principal e método \texttt{main};
                    \item Formas de declarar/instanciar variáveis e vetores;
                    \item Ciclo de vida das variáveis;
                    \item Referência e cópia;
                    \item Funcionamento do método construtor.
                \end{itemize}
            \item Criar um projeto simples para mostrar o uso das estruturas
                básicas (comparação, seleção, repetição, funções\ldots) da
                programação imperativa;
            \item Melhorar o desempenho e entendimento dos alunos nas
                disciplinas de Programação Orientada a Objetos 1 e 2;
            \item Instruir os participantes a respeito das boas práticas de
                programação, a fim de que façam códigos bem estruturados,
                reutilizáveis e legíveis, aplicando conceitos de código-limpo e
                que façam escolhas adequadas de padrões de projetos;
            \item Introduzir noções de algoritmos e resolução de
                problemas computacionalmente;
            \item Criar um projeto com enfoque no uso de diferentes recursos de
                programação orientada a objetos (interfaces, classes, classes
                abstratas, \ldots);
            \item Expor formas de melhorar a manutenibilidade de software.
        \end{itemize}
\end{description}

\section{Conteúdo Programático}

\begin{enumerate}
    \item Introdução a Java (2 horas):
        \begin{samepage}
            \begin{enumerate}
                \item Características principais da linguagem;
                \item Processo de compilação e \textit{Bytecode};
                \item Introdução à JVM (\textit{Java Virtual Machine});
                \item Introdução ao paradigma Imperativo;
                \item Tipos primitivos;
                \item Estruturas condicionais: \texttt{if}.
            \end{enumerate}
        \end{samepage}
    \item Coleções (2 horas):
        \begin{samepage}
            \begin{enumerate}
                \item Laços de repetição: \textit{while} e \textit{for};
                \item \textit{Raw-arrays};
                \item Tempo de vida de variáveis: Escopo;
                \item Matrizes;
                \item \textit{Cache-friendliness}.
            \end{enumerate}
        \end{samepage}
    \item Tipos definidos por usuário (2 horas):
        \begin{samepage}
            \begin{enumerate}
                \item Conceito de Classe, Objeto e Atributo;
                \item Alocação de memória na JVM\@;
                \item \textit{Garbage-Collection};
                \item Tempo de vida de objetos;
                \item Elementos estáticos.
            \end{enumerate}
        \end{samepage}
    \item Introdução a Algoritmos (2 horas):
        \begin{samepage}
            \begin{enumerate}
                \item Introdução a funções e procedimentos;
                \item Busca linear e binária;
                \item Complexidade de Algoritmos;
                \item Notação assintótica;
                \item \textit{Bubble-Sort}.
            \end{enumerate}
        \end{samepage}
    \item Métodos, funções e procedimentos (2 horas):
        \begin{samepage}
            \begin{enumerate}
                \item Cópia e Referência;
                \item Depuração de código;
                \item Encapsulamento de atributos;
                \item Boas práticas: API\@.
            \end{enumerate}
        \end{samepage}
    \item Padrões de Projeto (2 horas):
        \begin{samepage}
            \begin{enumerate}
                \item Diagrama de Casos de Uso;
                \item Padrão de projeto MVC\@;
                \item \textit{HashMap};
                \item Diagrama de Classes.
            \end{enumerate}
        \end{samepage}
    \item Persistência (2 horas):
        \begin{samepage}
            \begin{enumerate}
                \item \textit{Streams};
                \item Serialização;
                \item Disco rígido: \textit{Disk buffer};
                \item Algoritmos de escalonamento de disco;
                \item Tratamento de Exceções.
            \end{enumerate}
        \end{samepage}
    \item Introdução a Estruturas de Dados (2 horas):
        \begin{samepage}
            \begin{enumerate}
                \item \textit{Generics};
                \item Lista com vetor;
                \item Lista encadeada;
                \item Complexidade amortizada;
                \item Generics;
                \item Pilha;
                \item Padrão de projeto: Composição.
            \end{enumerate}
        \end{samepage}
    \item Outros aspectos da linguagem (2 horas):
        \begin{samepage}
            \begin{enumerate}
                \item Interfaces;
                \item Métodos \textit{default};
                \item Classes abstratas;
                \item \textit{JIT Compilation};
                \item Otimizações da JVM\@.
            \end{enumerate}
        \end{samepage}
\end{enumerate}

\section{Cronograma}

\begin{enumerate}[label= (\alph*)]
    \item Primeiro encontro:
        \begin{samepage}
            \begin{enumerate}
                \item Introdução a Java;
                \item Apresentação do material de apoio.
            \end{enumerate}
        \end{samepage}
    \item Segundo encontro:
        \begin{samepage}
            \begin{enumerate}
                \item Coleções.
            \end{enumerate}
        \end{samepage}
    \item Terceiro encontro:
        \begin{samepage}
            \begin{enumerate}
                \item Tipos definidos por usuário.
            \end{enumerate}
        \end{samepage}
    \item Quarto encontro:
        \begin{samepage}
            \begin{enumerate}
                \item Introdução a Algoritmos.
            \end{enumerate}
        \end{samepage}
    \item Quinto encontro:
        \begin{samepage}
            \begin{enumerate}
                \item Métodos, funções e procedimentos.
            \end{enumerate}
        \end{samepage}
    \item Sexto encontro:
        \begin{samepage}
            \begin{enumerate}
                \item Padrões de Projeto;
                \item Menção a padrões de projeto não abordados na aula.
            \end{enumerate}
        \end{samepage}
    \item Sétimo encontro:
        \begin{samepage}
            \begin{enumerate}
                \item Persistência.
            \end{enumerate}
        \end{samepage}
    \item Oitavo encontro:
        \begin{samepage}
            \begin{enumerate}
                \item Introdução a Estruturas de Dados;
            \end{enumerate}
        \end{samepage}
    \item Nono encontro:
        \begin{samepage}
            \begin{enumerate}
                \item Outros aspectos da linguagem;
                \item Conclusões e menções a respeito do criticismo de Java.
            \end{enumerate}
        \end{samepage}
\end{enumerate}

\nocite{*}
\bibliographystyle{unsrt}
\bibliography{plan}

\end{document}
