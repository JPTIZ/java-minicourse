\documentclass{article}

\usepackage[utf8]{inputenc}
\usepackage[T1]{fontenc}
\usepackage[portuguese]{babel}

\usepackage[inline]{enumitem}

\usepackage{listing}

\title{Programa do Minicurso}
\author{}
\date{}

\begin{document}

\maketitle

\section{Identificação}

\begin{itemize}
    \item[] Disciplina: Minicurso de Java Moderno
    \item[]
        \hspace{-1em}
        \begin{tabular}{ccc}
            Carga Horária: 20 horas-aula&
            Teóricas: 10&
            Práticas: 10
        \end{tabular}
    \item[] Período: Abril a Junho de 2017.
\end{itemize}

\section{Cursos (público alvo)}
\begin{itemize}
    \item[] Ciências da Computação;
    \item[] Sistemas de Informação.
\end{itemize}

\section{Requisitos}
\begin{itemize}
    \item[] Nenhum.
\end{itemize}

\section{Ementa}
\begin{itemize*}[label={}]
    % Básico
     \item Aritmética em programação; %OK%
     \item Estruturas Condicionais e de Seleção; %OK - Falta Seleção%
     \item Estruturas de Repetição; %OK%
     \item Métodos, Procedimentos e Funções; %OK%
    % Orientação a Objetos
     \item Encapsulamento de atributos; %OK%
     \item Interfaces; %OK%
     \item Tipos definidos por usuário; %OK%
     \item Classes abstratas; %OK%
    % Engenharia de Software
     \item Padrões de Projeto; %OK%
     \item Diagramas UML (\textit{Unified Modeling Language}): Classes
         e Casos de Uso; %OK%
    % GUI
     \item Componentes básicos AWT (\textit{Abstract Window Toolkit}); %OK%
     \item Padrão Observador-Observável; %OK%
     \item Padrão MVC\@; %OK%
    % Mídia
     \item \textit{Bitmaps}; %OK%
     \item \textit{Double-Buffering}; %OK%
     \item \textit{Page-Flipping}; %OK%
    % Geral
     \item \textit{Callback}; %OK%
     \item Reuso; %OK%
     \item Boas Práticas de Programação (padrões de nomenclatura/casing, etc.); %OK%
     \item Funções \textit{lambda}; %OK%
    % Otimização
     \item \textit{Cache friendliness}; %OK%
    % Java-specific
     \item \textit{Generics}; %OK%
     \item Classe Anônima; %OK%
     \item Mecanismo de alocação de memória da JVM\@; %OK%
    % Eclipse
     \item Introdução à ferramenta de desenvolvimento Eclipse; %OK%
     \item Depuração; %OK%
    % Ciência da Computação
     \item Análise de Algoritmos e Complexidade Assintótica; %OK%
    % Limbo of decisions
     \item \textit{HashMap}; %OK%
     \item \textit{Streams}; %OK%
     \item Serialização. %OK%
\end{itemize*}

\section{Objetivos}
\begin{description}
    \item[Geral:] Auxiliar na aprendizagem das disciplinas de
        Programação Orientada a Objetos 1 e 2, produzindo um projeto em
        console Java, seguido de sua adaptação para interface gráfica,
        mostrando normas e passos para um código bem feito e
        reutilizável.
    \item[Específico:] São objetivos do minicurso:
        \begin{itemize}[label={-}]
            \item Apresentar as especificidades básicas da linguagem
                Java, dentre elas:
                \begin{itemize}[label={-}]
                    \item Classe principal e método \texttt{main};
                    \item Formas de declarar/instanciar variáveis e
                        vetores;
                    \item Ciclo de vida das variáveis;
                    \item Referência e cópia;
                    \item Funcionamento do método construtor.
                \end{itemize}
            \item Criar um projeto simples para mostrar o uso das
                estruturas básicas (comparação, seleção, repetição,
                funções\ldots) da programação imperativa;
            \item Melhorar o desempenho e entendimento dos alunos nas
                disciplinas de Programação Orientada a Objetos 1 e 2;
            \item Instruir os participantes a respeito das boas
                práticas de programação, a fim de que façam códigos bem
                estruturados, reutilizáveis e legíveis, aplicando
                conceitos de código-limpo e que façam escolhas
                adequadas de padrões de projetos;
            \item Introduzir noções de algoritmos e resolução de
                problemas computacionalmente;
            \item Criar um projeto com enfoque no interfaceamento
                gráfico e uso de diferentes recursos de programação
                orientada a objetos (interfaces, classes, classes
                abstratas\ldots);
            \item Expor formas de melhorar a manutenibilidade de
                software.
        \end{itemize}
\end{description}

\section{Conteúdo Programático}

\begin{enumerate}
    \item Introdução a Java (2 horas):
        \begin{enumerate}
            \item Características principais da linguagem;
            \item Processo de compilação e \textit{Bytecode};
            \item Processo de execução da JVM (\textit{Java Virtual Machine});
            \item Introdução ao paradigma Imperativo;
            \item Tipos primitivos;
            \item Estruturas condicionais: \texttt{if}.
        \end{enumerate}
    \item Coleções (2 horas):
        \begin{enumerate}
            \item Laços de repetição;
            \item \textit{Raw-arrays};
            \item Tempo de vida de variáveis: Escopo;
            \item Matrizes;
            \item \textit{Cache-friendliness}.
        \end{enumerate}
    \item Tipos definidos por usuário (2 horas):
        \begin{enumerate}
            \item Classes e objetos;
            \item Atributos;
            \item Alocação de memória na JVM\@;
            \item Garbage-Collection;
            \item Tempo de vida de variáveis: Objetos;
            \item Elementos estáticos.
        \end{enumerate}
    \item Introdução a Algoritmos (2 horas):
        \begin{enumerate}
            \item Introdução a funções e procedimentos;
            \item Busca linear e \textit{Binary-Search};
            \item Complexidade de Algoritmos;
            \item Notação assintótica;
            \item \textit{Bubble-Sort}.
        \end{enumerate}
    \item Métodos, funções e procedimentos (2 horas):
        \begin{enumerate}
            \item Cópia e Referência;
            \item Depuração de código;
            \item Encapsulamento de atributos;
            \item Boas práticas: API\@.
        \end{enumerate}
    \item Padrões de Projeto (2 horas):
        \begin{enumerate}
            \item Diagrama de Casos de Uso;
            \item Padrão MVC\@;
            \item \textit{HashMap};
            \item Diagrama de Classes.
        \end{enumerate}
    \item Persistência (2 horas):
        \begin{enumerate}
            \item \textit{Streams};
            \item Serialização;
            \item Dependência entre objetos.
        \end{enumerate}
    \item Interface Gráfica (2 horas):
        \begin{enumerate}
            \item Componentes básicos AWT/\texttt{javax.swing};
            \item \textit{GridLayout};
            \item Padrão Observador-Observável;
            \item \textit{Callback};
            \item Funções \textit{lambda};
            \item Classes anônimas.
        \end{enumerate}
    \item Gráficos e renderização (2 horas):
        \begin{enumerate}
            \item Enumeradores;
            \item Estrutura de seleção;
            \item Cores RGB\@;
            \item \textit{Bitmaps};
            \item \textit{Double-Buffering};
            \item \textit{Page-Flipping}.
        \end{enumerate}
    \item Introdução a Estruturas de Dados (2 horas):
        \begin{enumerate}
            \item \textit{Generics};
            \item Lista com vetor;
            \item Lista encadeada;
            \item Interfaces;
            \item Classes abstratas;
            \item Pilha;
        \end{enumerate}
\end{enumerate}

\section{Cronograma}

\begin{enumerate}[label= (\alph*)]
    \item Primeiro encontro:
        \begin{enumerate}
            \item Introdução a Java;
            \item Apresentação do material de apoio.
        \end{enumerate}
    \item Segundo encontro:
        \begin{enumerate}
            \item Coleções.
        \end{enumerate}
    \item Terceiro encontro:
        \begin{enumerate}
            \item Tipos definidos por usuário.
        \end{enumerate}
    \item Quarto encontro:
        \begin{enumerate}
            \item Introdução a Algoritmos.
        \end{enumerate}
    \item Quinto encontro:
        \begin{enumerate}
            \item Métodos, funções e procedimentos.
        \end{enumerate}
    \item Sexto encontro:
        \begin{enumerate}
            \item Padrões de Projeto;
            \item Menção a padrões de projeto não abordados na aula.
        \end{enumerate}
    \item Sétimo encontro:
        \begin{enumerate}
            \item Persistência.
        \end{enumerate}
    \item Oitavo encontro:
        \begin{enumerate}
            \item Interface Gráfica.
        \end{enumerate}
    \item Nono encontro:
        \begin{enumerate}
            \item Gráficos e renderização.
        \end{enumerate}
    \item Décimo encontro:
        \begin{enumerate}
            \item Introdução a Estruturas de Dados;
            \item Conclusões e menções a respeito do criticismo de Java.
        \end{enumerate}
\end{enumerate}

\nocite{*}
\bibliographystyle{unsrt}
\bibliography{plan}

\end{document}
