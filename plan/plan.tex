\documentclass{article}

\usepackage[utf8]{inputenc}
\usepackage[T1]{fontenc}
\usepackage[portuguese]{babel}

\usepackage[inline]{enumitem}

\usepackage{hyperref}
\usepackage{listing}

\title{Programa do Minicurso}
\author{}
\date{}

\begin{document}

\maketitle

\section{Identificação}

\begin{itemize}
    \item[] Disciplina: Minicurso De Java Moderno
    \item[]
        \hspace{-1em}
        \begin{tabular}{ccc}
            Carga Horária: 20 horas-aula&
            Teóricas: 10&
            Práticas: 10
        \end{tabular}
    \item[] Período: Março a Junho de
                     2018.
\end{itemize}

\section{Cursos (público alvo)}
\begin{itemize}
    \item[] Ciência da Computação;
    \item[] Sistemas de Informação;
    \item[] Interessados na linguagem Java.
\end{itemize}

\section{Requisitos}
\begin{itemize}
    \item[] Nenhum.
\end{itemize}

\section{Ementa}
\begin{samepage}
\begin{itemize*}[label={}]
    \item[] Características principais da linguagem;
    \item[] \textit{Bytecode} e Processo de compilação;
    \item[] Introdução ao paradigma imperativo;
    \item[] Tipos primitivos;
    \item[] Inferência de tipo;
    \item[] Estruturas condicionais: \texttt{if}.
    \item[] Laço de repetição \textit{for};
    \item[] \textit{Raw-Arrays};
    \item[] Tempo de vida de variáveis: Escopo;
    \item[] \textit{For-each}.
    \item[] Introdução a funções e procedimentos;
    \item[] Busca linear;
    \item[] Complexidade de Algoritmos;
    \item[] Notação assintótica;
    \item[] \textit{Bubble-Sort}.
    \item[] Classe, Objeto e Atributo;
    \item[] Alocação de memória na JVM;
    \item[] \textit{Garbage-Collection};
    \item[] Tempo de vida de objetos.
    \item[] Método Construtor;
    \item[] Cópia e Referência;
    \item[] Boas práticas: API\@;
    \item[] Documentação de código.
    \item[] Processamento de texto;
    \item[] Laço de repetição: \textit{while};
    \item[] Introdução a segurança;
    \item[] Formato JSON;
    \item[] Operações em arquivos.
    \item[] Interfaces;
    \item[] Métodos \textit{default};
    \item[] Classes abstratas;
    \item[] Herança e problema do diamante;
    \item[] \textit{Generics}.
    \item[] \textit{Callback};
    \item[] Tipos de dados para funções;
    \item[] Funções \textit{lambda}.
    \item[] Versionamento Git;
    \item[] Sistema de pacotes Java;
    \item[] \textit{Classpath};
    \item[] Arquivo JAR\@;
    \item[] Arquivo JAR executável.
    \item[] Lista com vetor;
    \item[] Lista encadeada;
    \item[] Complexidade amortizada;
    \item[] Pilha;
    \item[] Padrão de projeto: Composição.
    \item[] Passos de otimização.
    \item[] \textit{Tail-call recursion}.
    \item[] \textit{JIT Compilation}.
    \item[] Otimizações da JVM\@.
\end{itemize*}
\end{samepage}

\section{Objetivos}

\begin{description}
    \item[Geral:] Auxiliar na aprendizagem das disciplinas de Programação Orientada a Objetos 1 e 2, demonstrando, através da criação de um projeto de escolha dos alunos, os recursos da linguagem, boas práticas, reusabilidade, \textit{tradeoffs} entre soluções (performance, uso de memória, manutenção...) e as novidades nas versões mais recentes..
    \item[Específico:] São objetivos do minicurso:
        \begin{itemize}[label={-}]
                \item Apresentar os recursos básicos da linguagem Java, dentre eles:
\begin{itemize}[label={-}]    \item Classes e método \texttt{main};
    \item Declaração/instanciação de variáveis;
    \item Ciclo de vida de variáveis;
    \item Referência e cópia;
    \item Métodos e construtores.
\end{itemize}Além das características e funcionamento interno deles;
                \item Criar um projeto simples para mostrar o uso das  estruturas básicas da programação imperativa  (comparação, seleção, repetição, funções\ldots);
                \item Melhorar o desempenho e entendimento dos alunos das  disciplinas de Programação Orientada a Objetos 1 e 2;
                \item Instruir os participantes a respeito das boas práticas de programação, a fim de que façam códigos bem  estruturados, reutilizáveis e legíveis, aplicando  conceitos de código-limpo e façam escolhas adequadas de  padrões de projetos;
                \item Introduzir noções de algoritmos e resolução de problemas computacionalmente;
                \item Criar um projeto com enfoque no uso de diferentes recursos de programação orientada a objetos (interfaces, classes, tipos abstratos, \ldots);
                \item Expor formas de melhorar a manutenibilidade de software através de metodologias da programação moderna;
                \item Demonstrar o uso dos recursos recentes e progressão da linguagem Java;
                \item Demonstrar o funcionamento interno da JVM (\textit{Java Virtual Machine}).
        \end{itemize}
\end{description}

\section{Conteúdo Programático}

\begin{enumerate}
    \item Introdução a Java (2 horas):
        \begin{samepage}
        \begin{enumerate}
                \item Características principais da linguagem;
                \item \textit{Bytecode} e Processo de compilação;
                \item Introdução ao paradigma imperativo;
                \item Tipos primitivos;
                \item Inferência de tipo;
                \item Estruturas condicionais: \texttt{if}.
        \end{enumerate}
        \end{samepage}
    \item Coleções (2 horas):
        \begin{samepage}
        \begin{enumerate}
                \item Laço de repetição \textit{for};
                \item \textit{Raw-Arrays};
                \item Tempo de vida de variáveis: Escopo;
                \item \textit{For-each}.
        \end{enumerate}
        \end{samepage}
    \item Introdução a Algoritmos (2 horas):
        \begin{samepage}
        \begin{enumerate}
                \item Introdução a funções e procedimentos;
                \item Busca linear;
                \item Complexidade de Algoritmos;
                \item Notação assintótica;
                \item \textit{Bubble-Sort}.
        \end{enumerate}
        \end{samepage}
    \item Tipos definidos por usuário (2 horas):
        \begin{samepage}
        \begin{enumerate}
                \item Classe, Objeto e Atributo;
                \item Alocação de memória na JVM;
                \item \textit{Garbage-Collection};
                \item Tempo de vida de objetos.
        \end{enumerate}
        \end{samepage}
    \item Métodos (2 horas):
        \begin{samepage}
        \begin{enumerate}
                \item Método Construtor;
                \item Cópia e Referência;
                \item Boas práticas: API\@;
                \item Documentação de código.
        \end{enumerate}
        \end{samepage}
    \item Persistência (2 horas):
        \begin{samepage}
        \begin{enumerate}
                \item Processamento de texto;
                \item Laço de repetição: \textit{while};
                \item Introdução a segurança;
                \item Formato JSON;
                \item Operações em arquivos.
        \end{enumerate}
        \end{samepage}
    \item Programação genérica e polimorfismo (2 horas):
        \begin{samepage}
        \begin{enumerate}
                \item Interfaces;
                \item Métodos \textit{default};
                \item Classes abstratas;
                \item Herança e problema do diamante;
                \item \textit{Generics}.
        \end{enumerate}
        \end{samepage}
    \item API funcional (2 horas):
        \begin{samepage}
        \begin{enumerate}
                \item \textit{Callback};
                \item Tipos de dados para funções;
                \item Funções \textit{lambda}.
        \end{enumerate}
        \end{samepage}
    \item Pacotes e bibliotecas (2 horas):
        \begin{samepage}
        \begin{enumerate}
                \item Versionamento Git;
                \item Sistema de pacotes Java;
                \item \textit{Classpath};
                \item Arquivo JAR\@;
                \item Arquivo JAR executável.
        \end{enumerate}
        \end{samepage}
    \item Introdução a Estruturas de Dados (2 horas):
        \begin{samepage}
        \begin{enumerate}
                \item Lista com vetor;
                \item Lista encadeada;
                \item Complexidade amortizada;
                \item Pilha;
                \item Padrão de projeto: Composição.
        \end{enumerate}
        \end{samepage}
    \item (Extra) Otimização (2 horas):
        \begin{samepage}
        \begin{enumerate}
                \item Passos de otimização;
                \item \textit{Tail-call recursion};
                \item \textit{JIT Compilation};
                \item Otimizações da JVM\@.
        \end{enumerate}
        \end{samepage}
\end{enumerate}

\section{Cronograma}

O cronograma segue a mesma ordenação do Conteúdo Programático, sendo cada item
uma aula separada.

\nocite{*}
\bibliographystyle{unsrt}
\bibliography{plan}

\end{document}